
% Default to the notebook output style

    


% Inherit from the specified cell style.




    
\documentclass[11pt]{article}

    
    
    \usepackage[T1]{fontenc}
    % Nicer default font (+ math font) than Computer Modern for most use cases
    \usepackage{mathpazo}

    % Basic figure setup, for now with no caption control since it's done
    % automatically by Pandoc (which extracts ![](path) syntax from Markdown).
    \usepackage{graphicx}
    % We will generate all images so they have a width \maxwidth. This means
    % that they will get their normal width if they fit onto the page, but
    % are scaled down if they would overflow the margins.
    \makeatletter
    \def\maxwidth{\ifdim\Gin@nat@width>\linewidth\linewidth
    \else\Gin@nat@width\fi}
    \makeatother
    \let\Oldincludegraphics\includegraphics
    % Set max figure width to be 80% of text width, for now hardcoded.
    \renewcommand{\includegraphics}[1]{\Oldincludegraphics[width=.8\maxwidth]{#1}}
    % Ensure that by default, figures have no caption (until we provide a
    % proper Figure object with a Caption API and a way to capture that
    % in the conversion process - todo).
    \usepackage{caption}
    \DeclareCaptionLabelFormat{nolabel}{}
    \captionsetup{labelformat=nolabel}

    \usepackage{adjustbox} % Used to constrain images to a maximum size 
    \usepackage{xcolor} % Allow colors to be defined
    \usepackage{enumerate} % Needed for markdown enumerations to work
    \usepackage{geometry} % Used to adjust the document margins
    \usepackage{amsmath} % Equations
    \usepackage{amssymb} % Equations
    \usepackage{textcomp} % defines textquotesingle
    % Hack from http://tex.stackexchange.com/a/47451/13684:
    \AtBeginDocument{%
        \def\PYZsq{\textquotesingle}% Upright quotes in Pygmentized code
    }
    \usepackage{upquote} % Upright quotes for verbatim code
    \usepackage{eurosym} % defines \euro
    \usepackage[mathletters]{ucs} % Extended unicode (utf-8) support
    \usepackage[utf8x]{inputenc} % Allow utf-8 characters in the tex document
    \usepackage{fancyvrb} % verbatim replacement that allows latex
    \usepackage{grffile} % extends the file name processing of package graphics 
                         % to support a larger range 
    % The hyperref package gives us a pdf with properly built
    % internal navigation ('pdf bookmarks' for the table of contents,
    % internal cross-reference links, web links for URLs, etc.)
    \usepackage{hyperref}
    \usepackage{longtable} % longtable support required by pandoc >1.10
    \usepackage{booktabs}  % table support for pandoc > 1.12.2
    \usepackage[inline]{enumitem} % IRkernel/repr support (it uses the enumerate* environment)
    \usepackage[normalem]{ulem} % ulem is needed to support strikethroughs (\sout)
                                % normalem makes italics be italics, not underlines
    

    
    
    % Colors for the hyperref package
    \definecolor{urlcolor}{rgb}{0,.145,.698}
    \definecolor{linkcolor}{rgb}{.71,0.21,0.01}
    \definecolor{citecolor}{rgb}{.12,.54,.11}

    % ANSI colors
    \definecolor{ansi-black}{HTML}{3E424D}
    \definecolor{ansi-black-intense}{HTML}{282C36}
    \definecolor{ansi-red}{HTML}{E75C58}
    \definecolor{ansi-red-intense}{HTML}{B22B31}
    \definecolor{ansi-green}{HTML}{00A250}
    \definecolor{ansi-green-intense}{HTML}{007427}
    \definecolor{ansi-yellow}{HTML}{DDB62B}
    \definecolor{ansi-yellow-intense}{HTML}{B27D12}
    \definecolor{ansi-blue}{HTML}{208FFB}
    \definecolor{ansi-blue-intense}{HTML}{0065CA}
    \definecolor{ansi-magenta}{HTML}{D160C4}
    \definecolor{ansi-magenta-intense}{HTML}{A03196}
    \definecolor{ansi-cyan}{HTML}{60C6C8}
    \definecolor{ansi-cyan-intense}{HTML}{258F8F}
    \definecolor{ansi-white}{HTML}{C5C1B4}
    \definecolor{ansi-white-intense}{HTML}{A1A6B2}

    % commands and environments needed by pandoc snippets
    % extracted from the output of `pandoc -s`
    \providecommand{\tightlist}{%
      \setlength{\itemsep}{0pt}\setlength{\parskip}{0pt}}
    \DefineVerbatimEnvironment{Highlighting}{Verbatim}{commandchars=\\\{\}}
    % Add ',fontsize=\small' for more characters per line
    \newenvironment{Shaded}{}{}
    \newcommand{\KeywordTok}[1]{\textcolor[rgb]{0.00,0.44,0.13}{\textbf{{#1}}}}
    \newcommand{\DataTypeTok}[1]{\textcolor[rgb]{0.56,0.13,0.00}{{#1}}}
    \newcommand{\DecValTok}[1]{\textcolor[rgb]{0.25,0.63,0.44}{{#1}}}
    \newcommand{\BaseNTok}[1]{\textcolor[rgb]{0.25,0.63,0.44}{{#1}}}
    \newcommand{\FloatTok}[1]{\textcolor[rgb]{0.25,0.63,0.44}{{#1}}}
    \newcommand{\CharTok}[1]{\textcolor[rgb]{0.25,0.44,0.63}{{#1}}}
    \newcommand{\StringTok}[1]{\textcolor[rgb]{0.25,0.44,0.63}{{#1}}}
    \newcommand{\CommentTok}[1]{\textcolor[rgb]{0.38,0.63,0.69}{\textit{{#1}}}}
    \newcommand{\OtherTok}[1]{\textcolor[rgb]{0.00,0.44,0.13}{{#1}}}
    \newcommand{\AlertTok}[1]{\textcolor[rgb]{1.00,0.00,0.00}{\textbf{{#1}}}}
    \newcommand{\FunctionTok}[1]{\textcolor[rgb]{0.02,0.16,0.49}{{#1}}}
    \newcommand{\RegionMarkerTok}[1]{{#1}}
    \newcommand{\ErrorTok}[1]{\textcolor[rgb]{1.00,0.00,0.00}{\textbf{{#1}}}}
    \newcommand{\NormalTok}[1]{{#1}}
    
    % Additional commands for more recent versions of Pandoc
    \newcommand{\ConstantTok}[1]{\textcolor[rgb]{0.53,0.00,0.00}{{#1}}}
    \newcommand{\SpecialCharTok}[1]{\textcolor[rgb]{0.25,0.44,0.63}{{#1}}}
    \newcommand{\VerbatimStringTok}[1]{\textcolor[rgb]{0.25,0.44,0.63}{{#1}}}
    \newcommand{\SpecialStringTok}[1]{\textcolor[rgb]{0.73,0.40,0.53}{{#1}}}
    \newcommand{\ImportTok}[1]{{#1}}
    \newcommand{\DocumentationTok}[1]{\textcolor[rgb]{0.73,0.13,0.13}{\textit{{#1}}}}
    \newcommand{\AnnotationTok}[1]{\textcolor[rgb]{0.38,0.63,0.69}{\textbf{\textit{{#1}}}}}
    \newcommand{\CommentVarTok}[1]{\textcolor[rgb]{0.38,0.63,0.69}{\textbf{\textit{{#1}}}}}
    \newcommand{\VariableTok}[1]{\textcolor[rgb]{0.10,0.09,0.49}{{#1}}}
    \newcommand{\ControlFlowTok}[1]{\textcolor[rgb]{0.00,0.44,0.13}{\textbf{{#1}}}}
    \newcommand{\OperatorTok}[1]{\textcolor[rgb]{0.40,0.40,0.40}{{#1}}}
    \newcommand{\BuiltInTok}[1]{{#1}}
    \newcommand{\ExtensionTok}[1]{{#1}}
    \newcommand{\PreprocessorTok}[1]{\textcolor[rgb]{0.74,0.48,0.00}{{#1}}}
    \newcommand{\AttributeTok}[1]{\textcolor[rgb]{0.49,0.56,0.16}{{#1}}}
    \newcommand{\InformationTok}[1]{\textcolor[rgb]{0.38,0.63,0.69}{\textbf{\textit{{#1}}}}}
    \newcommand{\WarningTok}[1]{\textcolor[rgb]{0.38,0.63,0.69}{\textbf{\textit{{#1}}}}}
    
    
    % Define a nice break command that doesn't care if a line doesn't already
    % exist.
    \def\br{\hspace*{\fill} \\* }
    % Math Jax compatability definitions
	\def\TeX{\mbox{T\kern-.14em\lower.5ex\hbox{E}\kern-.115em X}}
	\def\LaTeX{\mbox{L\kern-.325em\raise.21em\hbox{$\scriptstyle{A}$}\kern-.17em}\TeX}

    \def\gt{>}
    \def\lt{<}
    % Document parameters
    \title{Introduction and Python basics}
    
    
    

    % Pygments definitions
    
\makeatletter
\def\PY@reset{\let\PY@it=\relax \let\PY@bf=\relax%
    \let\PY@ul=\relax \let\PY@tc=\relax%
    \let\PY@bc=\relax \let\PY@ff=\relax}
\def\PY@tok#1{\csname PY@tok@#1\endcsname}
\def\PY@toks#1+{\ifx\relax#1\empty\else%
    \PY@tok{#1}\expandafter\PY@toks\fi}
\def\PY@do#1{\PY@bc{\PY@tc{\PY@ul{%
    \PY@it{\PY@bf{\PY@ff{#1}}}}}}}
\def\PY#1#2{\PY@reset\PY@toks#1+\relax+\PY@do{#2}}

\expandafter\def\csname PY@tok@gd\endcsname{\def\PY@tc##1{\textcolor[rgb]{0.63,0.00,0.00}{##1}}}
\expandafter\def\csname PY@tok@gu\endcsname{\let\PY@bf=\textbf\def\PY@tc##1{\textcolor[rgb]{0.50,0.00,0.50}{##1}}}
\expandafter\def\csname PY@tok@gt\endcsname{\def\PY@tc##1{\textcolor[rgb]{0.00,0.27,0.87}{##1}}}
\expandafter\def\csname PY@tok@gs\endcsname{\let\PY@bf=\textbf}
\expandafter\def\csname PY@tok@gr\endcsname{\def\PY@tc##1{\textcolor[rgb]{1.00,0.00,0.00}{##1}}}
\expandafter\def\csname PY@tok@cm\endcsname{\let\PY@it=\textit\def\PY@tc##1{\textcolor[rgb]{0.25,0.50,0.50}{##1}}}
\expandafter\def\csname PY@tok@vg\endcsname{\def\PY@tc##1{\textcolor[rgb]{0.10,0.09,0.49}{##1}}}
\expandafter\def\csname PY@tok@vi\endcsname{\def\PY@tc##1{\textcolor[rgb]{0.10,0.09,0.49}{##1}}}
\expandafter\def\csname PY@tok@vm\endcsname{\def\PY@tc##1{\textcolor[rgb]{0.10,0.09,0.49}{##1}}}
\expandafter\def\csname PY@tok@mh\endcsname{\def\PY@tc##1{\textcolor[rgb]{0.40,0.40,0.40}{##1}}}
\expandafter\def\csname PY@tok@cs\endcsname{\let\PY@it=\textit\def\PY@tc##1{\textcolor[rgb]{0.25,0.50,0.50}{##1}}}
\expandafter\def\csname PY@tok@ge\endcsname{\let\PY@it=\textit}
\expandafter\def\csname PY@tok@vc\endcsname{\def\PY@tc##1{\textcolor[rgb]{0.10,0.09,0.49}{##1}}}
\expandafter\def\csname PY@tok@il\endcsname{\def\PY@tc##1{\textcolor[rgb]{0.40,0.40,0.40}{##1}}}
\expandafter\def\csname PY@tok@go\endcsname{\def\PY@tc##1{\textcolor[rgb]{0.53,0.53,0.53}{##1}}}
\expandafter\def\csname PY@tok@cp\endcsname{\def\PY@tc##1{\textcolor[rgb]{0.74,0.48,0.00}{##1}}}
\expandafter\def\csname PY@tok@gi\endcsname{\def\PY@tc##1{\textcolor[rgb]{0.00,0.63,0.00}{##1}}}
\expandafter\def\csname PY@tok@gh\endcsname{\let\PY@bf=\textbf\def\PY@tc##1{\textcolor[rgb]{0.00,0.00,0.50}{##1}}}
\expandafter\def\csname PY@tok@ni\endcsname{\let\PY@bf=\textbf\def\PY@tc##1{\textcolor[rgb]{0.60,0.60,0.60}{##1}}}
\expandafter\def\csname PY@tok@nl\endcsname{\def\PY@tc##1{\textcolor[rgb]{0.63,0.63,0.00}{##1}}}
\expandafter\def\csname PY@tok@nn\endcsname{\let\PY@bf=\textbf\def\PY@tc##1{\textcolor[rgb]{0.00,0.00,1.00}{##1}}}
\expandafter\def\csname PY@tok@no\endcsname{\def\PY@tc##1{\textcolor[rgb]{0.53,0.00,0.00}{##1}}}
\expandafter\def\csname PY@tok@na\endcsname{\def\PY@tc##1{\textcolor[rgb]{0.49,0.56,0.16}{##1}}}
\expandafter\def\csname PY@tok@nb\endcsname{\def\PY@tc##1{\textcolor[rgb]{0.00,0.50,0.00}{##1}}}
\expandafter\def\csname PY@tok@nc\endcsname{\let\PY@bf=\textbf\def\PY@tc##1{\textcolor[rgb]{0.00,0.00,1.00}{##1}}}
\expandafter\def\csname PY@tok@nd\endcsname{\def\PY@tc##1{\textcolor[rgb]{0.67,0.13,1.00}{##1}}}
\expandafter\def\csname PY@tok@ne\endcsname{\let\PY@bf=\textbf\def\PY@tc##1{\textcolor[rgb]{0.82,0.25,0.23}{##1}}}
\expandafter\def\csname PY@tok@nf\endcsname{\def\PY@tc##1{\textcolor[rgb]{0.00,0.00,1.00}{##1}}}
\expandafter\def\csname PY@tok@si\endcsname{\let\PY@bf=\textbf\def\PY@tc##1{\textcolor[rgb]{0.73,0.40,0.53}{##1}}}
\expandafter\def\csname PY@tok@s2\endcsname{\def\PY@tc##1{\textcolor[rgb]{0.73,0.13,0.13}{##1}}}
\expandafter\def\csname PY@tok@nt\endcsname{\let\PY@bf=\textbf\def\PY@tc##1{\textcolor[rgb]{0.00,0.50,0.00}{##1}}}
\expandafter\def\csname PY@tok@nv\endcsname{\def\PY@tc##1{\textcolor[rgb]{0.10,0.09,0.49}{##1}}}
\expandafter\def\csname PY@tok@s1\endcsname{\def\PY@tc##1{\textcolor[rgb]{0.73,0.13,0.13}{##1}}}
\expandafter\def\csname PY@tok@dl\endcsname{\def\PY@tc##1{\textcolor[rgb]{0.73,0.13,0.13}{##1}}}
\expandafter\def\csname PY@tok@ch\endcsname{\let\PY@it=\textit\def\PY@tc##1{\textcolor[rgb]{0.25,0.50,0.50}{##1}}}
\expandafter\def\csname PY@tok@m\endcsname{\def\PY@tc##1{\textcolor[rgb]{0.40,0.40,0.40}{##1}}}
\expandafter\def\csname PY@tok@gp\endcsname{\let\PY@bf=\textbf\def\PY@tc##1{\textcolor[rgb]{0.00,0.00,0.50}{##1}}}
\expandafter\def\csname PY@tok@sh\endcsname{\def\PY@tc##1{\textcolor[rgb]{0.73,0.13,0.13}{##1}}}
\expandafter\def\csname PY@tok@ow\endcsname{\let\PY@bf=\textbf\def\PY@tc##1{\textcolor[rgb]{0.67,0.13,1.00}{##1}}}
\expandafter\def\csname PY@tok@sx\endcsname{\def\PY@tc##1{\textcolor[rgb]{0.00,0.50,0.00}{##1}}}
\expandafter\def\csname PY@tok@bp\endcsname{\def\PY@tc##1{\textcolor[rgb]{0.00,0.50,0.00}{##1}}}
\expandafter\def\csname PY@tok@c1\endcsname{\let\PY@it=\textit\def\PY@tc##1{\textcolor[rgb]{0.25,0.50,0.50}{##1}}}
\expandafter\def\csname PY@tok@fm\endcsname{\def\PY@tc##1{\textcolor[rgb]{0.00,0.00,1.00}{##1}}}
\expandafter\def\csname PY@tok@o\endcsname{\def\PY@tc##1{\textcolor[rgb]{0.40,0.40,0.40}{##1}}}
\expandafter\def\csname PY@tok@kc\endcsname{\let\PY@bf=\textbf\def\PY@tc##1{\textcolor[rgb]{0.00,0.50,0.00}{##1}}}
\expandafter\def\csname PY@tok@c\endcsname{\let\PY@it=\textit\def\PY@tc##1{\textcolor[rgb]{0.25,0.50,0.50}{##1}}}
\expandafter\def\csname PY@tok@mf\endcsname{\def\PY@tc##1{\textcolor[rgb]{0.40,0.40,0.40}{##1}}}
\expandafter\def\csname PY@tok@err\endcsname{\def\PY@bc##1{\setlength{\fboxsep}{0pt}\fcolorbox[rgb]{1.00,0.00,0.00}{1,1,1}{\strut ##1}}}
\expandafter\def\csname PY@tok@mb\endcsname{\def\PY@tc##1{\textcolor[rgb]{0.40,0.40,0.40}{##1}}}
\expandafter\def\csname PY@tok@ss\endcsname{\def\PY@tc##1{\textcolor[rgb]{0.10,0.09,0.49}{##1}}}
\expandafter\def\csname PY@tok@sr\endcsname{\def\PY@tc##1{\textcolor[rgb]{0.73,0.40,0.53}{##1}}}
\expandafter\def\csname PY@tok@mo\endcsname{\def\PY@tc##1{\textcolor[rgb]{0.40,0.40,0.40}{##1}}}
\expandafter\def\csname PY@tok@kd\endcsname{\let\PY@bf=\textbf\def\PY@tc##1{\textcolor[rgb]{0.00,0.50,0.00}{##1}}}
\expandafter\def\csname PY@tok@mi\endcsname{\def\PY@tc##1{\textcolor[rgb]{0.40,0.40,0.40}{##1}}}
\expandafter\def\csname PY@tok@kn\endcsname{\let\PY@bf=\textbf\def\PY@tc##1{\textcolor[rgb]{0.00,0.50,0.00}{##1}}}
\expandafter\def\csname PY@tok@cpf\endcsname{\let\PY@it=\textit\def\PY@tc##1{\textcolor[rgb]{0.25,0.50,0.50}{##1}}}
\expandafter\def\csname PY@tok@kr\endcsname{\let\PY@bf=\textbf\def\PY@tc##1{\textcolor[rgb]{0.00,0.50,0.00}{##1}}}
\expandafter\def\csname PY@tok@s\endcsname{\def\PY@tc##1{\textcolor[rgb]{0.73,0.13,0.13}{##1}}}
\expandafter\def\csname PY@tok@kp\endcsname{\def\PY@tc##1{\textcolor[rgb]{0.00,0.50,0.00}{##1}}}
\expandafter\def\csname PY@tok@w\endcsname{\def\PY@tc##1{\textcolor[rgb]{0.73,0.73,0.73}{##1}}}
\expandafter\def\csname PY@tok@kt\endcsname{\def\PY@tc##1{\textcolor[rgb]{0.69,0.00,0.25}{##1}}}
\expandafter\def\csname PY@tok@sc\endcsname{\def\PY@tc##1{\textcolor[rgb]{0.73,0.13,0.13}{##1}}}
\expandafter\def\csname PY@tok@sb\endcsname{\def\PY@tc##1{\textcolor[rgb]{0.73,0.13,0.13}{##1}}}
\expandafter\def\csname PY@tok@sa\endcsname{\def\PY@tc##1{\textcolor[rgb]{0.73,0.13,0.13}{##1}}}
\expandafter\def\csname PY@tok@k\endcsname{\let\PY@bf=\textbf\def\PY@tc##1{\textcolor[rgb]{0.00,0.50,0.00}{##1}}}
\expandafter\def\csname PY@tok@se\endcsname{\let\PY@bf=\textbf\def\PY@tc##1{\textcolor[rgb]{0.73,0.40,0.13}{##1}}}
\expandafter\def\csname PY@tok@sd\endcsname{\let\PY@it=\textit\def\PY@tc##1{\textcolor[rgb]{0.73,0.13,0.13}{##1}}}

\def\PYZbs{\char`\\}
\def\PYZus{\char`\_}
\def\PYZob{\char`\{}
\def\PYZcb{\char`\}}
\def\PYZca{\char`\^}
\def\PYZam{\char`\&}
\def\PYZlt{\char`\<}
\def\PYZgt{\char`\>}
\def\PYZsh{\char`\#}
\def\PYZpc{\char`\%}
\def\PYZdl{\char`\$}
\def\PYZhy{\char`\-}
\def\PYZsq{\char`\'}
\def\PYZdq{\char`\"}
\def\PYZti{\char`\~}
% for compatibility with earlier versions
\def\PYZat{@}
\def\PYZlb{[}
\def\PYZrb{]}
\makeatother


    % Exact colors from NB
    \definecolor{incolor}{rgb}{0.0, 0.0, 0.5}
    \definecolor{outcolor}{rgb}{0.545, 0.0, 0.0}



    
    % Prevent overflowing lines due to hard-to-break entities
    \sloppy 
    % Setup hyperref package
    \hypersetup{
      breaklinks=true,  % so long urls are correctly broken across lines
      colorlinks=true,
      urlcolor=urlcolor,
      linkcolor=linkcolor,
      citecolor=citecolor,
      }
    % Slightly bigger margins than the latex defaults
    
    \geometry{verbose,tmargin=1in,bmargin=1in,lmargin=1in,rmargin=1in}
    
    

    \begin{document}
    
    
    \maketitle
    
    

    
    \section{Introduction to Scientific Computing with
Python}\label{introduction-to-scientific-computing-with-python}

    \subsection{Why Python?}\label{why-python}

\begin{itemize}
\item
  Wide set of scientific computing functionality
\item
  Numerical and scientific libraries
\item
  Plotting (graphs, charts, etc)
\item
  Interfacing to existing Fortran/C/C++ code, use Python as "glue".
\item
  Python is a high level language
\item
  Simple to learn and write
\item
  Spend time thinking about what code does rather than how to write it
\item
  Supports different programming styles: procedural or object-oriented
\end{itemize}

    \subsection{Core libraries and packages for scientific
computing}\label{core-libraries-and-packages-for-scientific-computing}

\paragraph{Standard library:}\label{standard-library}

\begin{itemize}
\tightlist
\item
  https://docs.python.org/3/library/
\end{itemize}

\paragraph{We will cover the central
packages:}\label{we-will-cover-the-central-packages}

\begin{itemize}
\item
  NumPy
\item
  Tools for manipulating arrays
\item
  Matplotlib
\item
  Plotting in 2D and 3D
\item
  SciPy
\item
  Higher-level scientific routines for common algorithms e.g. numerical
  integration, optimisation, Fourier transforms
\end{itemize}

    \subsection{Other useful packages}\label{other-useful-packages}

\begin{itemize}
\item
  numba
\item
  Decorations to allow compilation to source
\item
  pandas
\item
  Data structures and analysis
\item
  scikit-learn
\item
  Machine learning, data mining and data analysis
\item
  bokeh
\item
  Interactive visualization library
\end{itemize}

See * http://pythonhosted.org/mpi4py/ * http://numba.pydata.org *
http://pandas.pydata.org * http://scikit-learn.org *
http://bokeh.pydata.org

    \subsection{Scientific computing
overview}\label{scientific-computing-overview}

\subsubsection{A typical workflow}\label{a-typical-workflow}

\begin{itemize}
\item
  Generate data
\item
  Perhaps from simulation on HPC facilities
\item
  Perhaps from experiment
\item
  Process data
\item
  Compute/extract appropriate results from data
\item
  Visualise results
\item
  To understand the significance of our work and gain scientific insight
\item
  Communicate results
\item
  Through publications, presentations, web, etc.
\end{itemize}

    \begin{Verbatim}[commandchars=\\\{\}]
{\color{incolor}In [{\color{incolor}2}]:} \PY{n+nb}{print} \PY{p}{(}\PY{l+s+s2}{\PYZdq{}}\PY{l+s+s2}{Hello this is working}\PY{l+s+s2}{\PYZdq{}}\PY{p}{)}
\end{Verbatim}


    \begin{Verbatim}[commandchars=\\\{\}]
Hello this is working

    \end{Verbatim}

    \subsection{How to use Python?}\label{how-to-use-python}

 * Python code is executed by the Python interpreter:
\textbf{\texttt{python}}

\begin{itemize}
\item
  \textbf{Interactive mode} run Python interpreter without an input
  script file
\item
  Interpreter runs as a Python shell (interactive Python runtime
  environment)
\item
  \textbf{Non-interactive mode} : supply the Python interpreter with an
  input script file
\item
  \texttt{python\ myscript.py}
\item
  \textbf{Alternative interactive modes}
\item
  IPython : enchanced Python shell, ideal for data manipulation and
  visualisation
\item
  Jupyter (formerly IPython) notebook : web browser-based interactive
  document
\end{itemize}

    \subsection{Jupyter Notebook}\label{jupyter-notebook}

A browser interface to the iPython shell

\begin{itemize}
\tightlist
\item
  Input is split into cells
\item
  So far we have seen "Markdown" cells
\item
  Enter python commands into "code" cells
\end{itemize}

    \subsection{Python basics recap}\label{python-basics-recap}

\subsubsection{Data types}\label{data-types}

\begin{itemize}
\tightlist
\item
  Python is "dynamically typed": no explicit type declarations
\item
  Data type worked out from context when code is executed
\item
  Basic types include integers, floating point numbers, strings...
\end{itemize}

    \subsection{Python basics recap}\label{python-basics-recap}

\paragraph{Dynamic typing requires some
care}\label{dynamic-typing-requires-some-care}

    \begin{Verbatim}[commandchars=\\\{\}]
{\color{incolor}In [{\color{incolor}4}]:} \PY{n}{a} \PY{o}{=} \PY{l+m+mi}{3}
        \PY{n}{b} \PY{o}{=} \PY{l+m+mi}{4}
        \PY{n}{c} \PY{o}{=} \PY{l+s+s2}{\PYZdq{}}\PY{l+s+s2}{5}\PY{l+s+s2}{\PYZdq{}}
        
        \PY{n+nb}{print} \PY{p}{(}\PY{n}{a} \PY{o}{+} \PY{n}{b}\PY{p}{)}
        \PY{n+nb}{print} \PY{p}{(}\PY{n}{b} \PY{o}{+} \PY{n+nb}{int}\PY{p}{(}\PY{n}{c}\PY{p}{)}\PY{p}{)}
\end{Verbatim}


    \begin{Verbatim}[commandchars=\\\{\}]
7
9

    \end{Verbatim}

    \subsection{Python basics recap}\label{python-basics-recap}

\subsubsection{Data structures}\label{data-structures}

\begin{itemize}
\tightlist
\item
  lists, e.g., {[}3, "a", 3.14, False{]}
\item
  dictionary (or "dicts") \{"key1" : "value1", "key1" : "value1"\}
\item
  tuples (1,2,3) or (1.2,) or (1, )
\end{itemize}

A tuple is an example of an immutable object. Immutable means cannot be
changed

    \subsection{Python basics recap}\label{python-basics-recap}

 \#\#\# White space is significant (so take care!)

\begin{itemize}
\tightlist
\item
  Code blocks are indented

  \begin{itemize}
  \tightlist
  \item
    loops
  \item
    conditionals
  \item
    functions
  \end{itemize}
\item
  Indent should be exactly 4 spaces

  \begin{itemize}
  \tightlist
  \item
    not 3 spaces, not 5 spaces, not a tab stop, ... 
  \end{itemize}
\end{itemize}

    \subsection{Python basics recap}\label{python-basics-recap}

\subsubsection{Importing additional
modules}\label{importing-additional-modules}

\begin{itemize}
\item
  There are a number of options:
\item
  \texttt{import\ module}
\item
  \texttt{from\ module\ import\ name}
\item
  avoid universal imports
\item
  E.g., from the standard library
\item
  https://docs.python.org/2/library/
\end{itemize}

    For example, the standard library contains a module \texttt{random} for
the generation of random numbers. The module contains a function
\texttt{random()}, amongst others.

    \subsection{Python basics recap}\label{python-basics-recap}

\begin{itemize}
\item
  Functions
\item
  In-built functions, e.g., \texttt{int()}, \texttt{type()},
  \texttt{range()}
\item
  Object methods are accessed with dot operator :
  \texttt{object.method()} e.g. \texttt{list.sort()}
\item
  Module functions also accessed with dot operator :
  \texttt{module.function()}
\item
  Define your own function, e.g.:

\begin{Shaded}
\begin{Highlighting}[]
\KeywordTok{def} \NormalTok{my_square(x):}
    \ControlFlowTok{return} \NormalTok{x}\OperatorTok{*}\NormalTok{x}
\end{Highlighting}
\end{Shaded}
\end{itemize}

See * In-built functions
https://docs.python.org/2/library/functions.html

    \subsection{A Python module}\label{a-python-module}

\begin{itemize}
\item
  A file with extension \texttt{.py} is a "module"
\item
  At the top of such a file you may see something like:

\begin{Shaded}
\begin{Highlighting}[]
\CommentTok{#!/usr/bin/env python}
\end{Highlighting}
\end{Shaded}

  (Unix) If the file is executable, locate interpreter from environment
  \texttt{\$PATH}
\item
  At the bottom, you may see

\begin{Shaded}
\begin{Highlighting}[]
\ControlFlowTok{if} \VariableTok{__name__} \OperatorTok{==} \StringTok{"__main__"}\NormalTok{:}
\ImportTok{import} \NormalTok{sys}
\NormalTok{some_function_defined_above(sys.argv)}
\end{Highlighting}
\end{Shaded}
\item
  Allows use as "stand-alone" script or as imported module 
\end{itemize}

    \subsection{Exercise 1: Writing a
function}\label{exercise-1-writing-a-function}

 \#\#\# Median from a list

Define a function my\_median() that takes a list of years and computes
the median based on the elapsed number of years since 1900 for each
entry in the list.

The function should return the median and the value either side of the
median as a list.

If \texttt{years\ =\ {[}1989,\ 1955,\ 2011,\ 1943,\ 1975{]}} then the
result should be

\begin{Shaded}
\begin{Highlighting}[]
\BuiltInTok{print} \NormalTok{my_median(years)}
\NormalTok{[}\DecValTok{55}\NormalTok{, }\DecValTok{75}\NormalTok{, }\DecValTok{89}\NormalTok{]}
\end{Highlighting}
\end{Shaded}

Assume the input list is not ordered and has an odd number of elements
\(N\), where \(N &gt;= 3\).

\begin{enumerate}
\def\labelenumi{\arabic{enumi}.}
\item
  Use the above list of numbers as input to check you get the right
  answer.
\item
  You will need to sort the list. How?
\item
  List indexing may help you to get the final result e.g.
  \texttt{list{[}3:5{]}}
\item
  Try generating a random array for input using module random.
\end{enumerate}

    \begin{Verbatim}[commandchars=\\\{\}]
{\color{incolor}In [{\color{incolor}9}]:} \PY{k}{def} \PY{n+nf}{my\PYZus{}median}\PY{p}{(}\PY{n}{years}\PY{p}{)}\PY{p}{:}
            \PY{n}{ages} \PY{o}{=} \PY{p}{[}\PY{p}{]}
            \PY{k}{for} \PY{n}{y} \PY{o+ow}{in} \PY{n}{years}\PY{p}{:}
                \PY{n}{ages}\PY{o}{.}\PY{n}{append}\PY{p}{(}\PY{n}{y} \PY{o}{\PYZhy{}} \PY{l+m+mi}{1990}\PY{p}{)}
            
            \PY{n}{ages}\PY{o}{.}\PY{n}{sort}\PY{p}{(}\PY{p}{)}
            
            \PY{n}{mid} \PY{o}{=} \PY{n+nb}{len}\PY{p}{(}\PY{n}{ages}\PY{p}{)}\PY{o}{/}\PY{o}{/}\PY{l+m+mi}{2}
            \PY{n+nb}{print} \PY{p}{(}\PY{n}{mid}\PY{p}{)}
            
            \PY{c+c1}{\PYZsh{}slicing}
            \PY{n}{result} \PY{o}{=} \PY{n}{ages}\PY{p}{[}\PY{n}{mid} \PY{o}{\PYZhy{}} \PY{l+m+mi}{1} \PY{p}{:} \PY{n}{mid} \PY{o}{+} \PY{l+m+mi}{1}\PY{p}{]}
            \PY{k}{return} \PY{n}{result}
            
        \PY{n}{years} \PY{o}{=} \PY{p}{[}\PY{l+m+mi}{1989}\PY{p}{,} \PY{l+m+mi}{1955}\PY{p}{,} \PY{l+m+mi}{2011}\PY{p}{,} \PY{l+m+mi}{1943}\PY{p}{,} \PY{l+m+mi}{1975}\PY{p}{]}
        \PY{n+nb}{print} \PY{p}{(}\PY{n}{my\PYZus{}median}\PY{p}{(}\PY{n}{years}\PY{p}{)}\PY{p}{)}
\end{Verbatim}


    \begin{Verbatim}[commandchars=\\\{\}]
2
[-35, -15]

    \end{Verbatim}

    \subsection{Exercise 2 : Functions and
libraries}\label{exercise-2-functions-and-libraries}

 \#\#\# Using an external file

Now read the list of years from a text file, years.txt, which should
have the total number of years \(N\) in the first line, followed by a
numbered list of years:

\begin{verbatim}
total number of years
1 year1
2 year2
...
\end{verbatim}

To generate an input file '\texttt{years.txt}', you can try code of the
form:

\begin{Shaded}
\begin{Highlighting}[]
\ImportTok{import} \NormalTok{sys}
\NormalTok{output }\OperatorTok{=} \BuiltInTok{open}\NormalTok{(filename, }\StringTok{"w"}\NormalTok{)}
\NormalTok{...}
\NormalTok{output.write(}\StringTok{"}\SpecialCharTok{\{0:2d\}}\StringTok{ }\SpecialCharTok{\{1:2d\}}\CharTok{\textbackslash{}n}\StringTok{"}\NormalTok{.}\BuiltInTok{format}\NormalTok{(}\DecValTok{1}\NormalTok{, years[}\DecValTok{0}\NormalTok{]))}
\end{Highlighting}
\end{Shaded}

To read an input file '\texttt{years.txt}', you may need code of the
form:

\begin{Shaded}
\begin{Highlighting}[]
\BuiltInTok{input} \OperatorTok{=} \BuiltInTok{open}\NormalTok{(filename, }\StringTok{"w"}\NormalTok{)}
\NormalTok{line }\OperatorTok{=} \BuiltInTok{input}\NormalTok{.readline()}
\NormalTok{line.rstrip()}
\NormalTok{tokens }\OperatorTok{=} \NormalTok{line.split()}
\end{Highlighting}
\end{Shaded}

Check the in-built or web documentation for help with these standard
library functions.

    

    \subsection{Summary}\label{summary}

 * We have reviewed some core Python basics

\begin{itemize}
\item
  We have also been introduced to the IPython shell
\item
  We will now look the packages that form the backbone of scientific
  python
\item
  NumPy (next)
\item
  Matplotlib
\item
  SciPy
\item
  MPI4PY
\end{itemize}

\begin{itemize}
\item
  Useful links
\item
  https://docs.python.org/2/
\item
  https://www.codecademy.com/learn/python
\end{itemize}


    % Add a bibliography block to the postdoc
    
    
    
    \end{document}
